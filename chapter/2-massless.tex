\chapter{Massless Medium}
%%%%%%%%%%%%%%%%%%%%%%%%%%%%%%%%%%%%%%%%%%%%%%%%%%%%%%%%%%%%%%%%%%%%%%%%%%%%%%%%%%%%%%%%%%%%%%%%%%%%%%%%%%%%%%%%%%%%
In this chapter I will consider one-loop contributions that involve a massless medium.
%%%%%%%%%%%%%%%%%%%%%%%%%%%%%%%%%%%%%%%%%%%%%%%%%%%%%%%%%%%%%%%%%%%%%%%%%%%%%%%%%%%%%%%%%%%%%%%%%%%%%%%%%%%%%%%%%%%%
\section{Box}
%%%%%%%%%%%%%%%%%%%%%%%%%%%%%%%%%%%%%%%%%%%%%%%%%%%%%%%%%%%%%%%%%%%%%%%%%%%%%%%%%%%%%%%%%%%%%%%%%%%%%%%%%%%%%%%%%%%%
The box correlator in a massless medium is:
\begin{equation}
	\mathcal{B}_{0}(x) = G_{0}(x_{1}| x_{2}) G_{\Phi}(x_{3}| x_{1}) G_{0}(x_{3}| x_{4}) G_{\Psi}(x_{4}| x_{2}).
	\label{eq:box_A}
\end{equation}
In terms of four Schwinger moduli you have
\begin{equation}
	\mathcal{B}_{0} = \int\limits_{0}^{\infty} \int\limits_{0}^{\infty} \int\limits_{0}^{\infty} \int\limits_{0}^{\infty} \mathrm{d}T_{12} \mathrm{d}T_{31} \mathrm{d}T_{34} \mathrm{d}T_{42} \left( \frac{1}{T_{12} T_{31} T_{34} T_{42}} \right)^{D/2} \exp{\left[- \frac{1}{2} B_{0}(x, T) \right]},
\end{equation}
where
\begin{equation}
	B_{0} = \frac{1}{T_{12}} \abs{x_{12}}^{2} + \frac{1}{T_{31}} \abs{x_{31}}^{2} + m_{1}^{2} T_{31} + \frac{1}{T_{34}} \abs{x_{34}}^{2} + \frac{1}{T_{42}} \abs{x_{42}}^{2} + m_{2}^{2} T_{42}.
\end{equation}
The box amplitude follows from the Fourier transform:
\begin{equation}
	\widehat{\mathcal{B}}_{0}(p) = \int \int \int \int \mathrm{d}x_{1} \mathrm{d}x_{2} \mathrm{d}x_{3} \mathrm{d}x_{4} \, \mathcal{B}_{0}(x) \exp{\left[ i \mathbb{F}(x, p) \right]},
\end{equation}
with $\mathbb{F}$ given by (\ref{eq:F}). Note that
\begin{equation}
	x_{34} - x_{31} - x_{12} + x_{42} = 0.
\end{equation}
That is,
\begin{equation}
	|x_{12}|^{2} = |x_{34} - x_{31} + x_{42}|^{2}.
\end{equation}
We make a change of variables:
\begin{equation}
	\mathrm{d}x_{1} \mathrm{d}x_{2} \mathrm{d}x_{3} \mathrm{d}x_{4} \sim \mathrm{d}X \mathrm{d}x_{34} \mathrm{d}x_{31} \mathrm{d}x_{42} = \int \mathrm{d}X \mathrm{d}x_{34} \mathrm{d}x_{31} \mathrm{d}x_{12} \mathrm{d}x_{42} \delta(x_{34} - x_{31} - x_{12} + x_{42}),
\end{equation}
and use
\begin{equation}
	\delta(x_{34} - x_{31} - x_{12} + x_{42}) = \int \mathrm{d}q \, \exp{\left[-i q \cdot (x_{34} - x_{31} - x_{12} + x_{42}) \right]},
\end{equation}
to perform the integration over the spacetime positions:
\begin{equation}
	\widehat{\mathcal{B}}_{0}(p) = \delta(P) \int \mathrm{d}q \int\limits_{0}^{\infty} \int\limits_{0}^{\infty} \int\limits_{0}^{\infty} \int\limits_{0}^{\infty} \mathrm{d}T_{12} \mathrm{d}T_{31} \mathrm{d}T_{34} \mathrm{d}T_{42} \, \exp{\left[- \frac{1}{2} \widehat{B}_{0}(p, q, T) \right]},
	\label{eq:hat_calB_A}
\end{equation}
where
\begin{equation}
	\widehat{B}_{0} = \abs{q}^{2} T_{12} + \left(\abs{q - p_{31}}^{2} + m_{1}^{2} \right) T_{31} + \abs{q - p_{34}}^{2} T_{34} + \left(\abs{q + p_{42}}^{2} + m_{2}^{2} \right) T_{42}.
\end{equation}
This result is kinematic-exact.
%%%%%%%%%%%%%%%%%%%%%%%%%%%%%%%%%%%%%%%%%%%%%%%%%%%%%%%%%%%%%%%%%%%%%%%%%%%%%%%%%%%%%%%%%%%%%%%%%%%%%%%%%%%%%%%%%%%%
\subsection{Sudakov Moduli}
%%%%%%%%%%%%%%%%%%%%%%%%%%%%%%%%%%%%%%%%%%%%%%%%%%%%%%%%%%%%%%%%%%%%%%%%%%%%%%%%%%%%%%%%%%%%%%%%%%%%%%%%%%%%%%%%%%%%
One can integrate over the Schwinger moduli in (\ref{eq:hat_calB_A}) to obtain:
\begin{equation}
	\widehat{\mathcal{B}}_{A}(p) = \delta(P) \int \mathrm{d}q \left( \frac{2}{\abs{q}^{2}} \right) \left( \frac{2}{\abs{q - p_{31}}^{2} + m_{1}^{2}} \right) \left( \frac{2}{\abs{q - p_{34}}^{2}} \right) \left( \frac{2}{\abs{q + p_{42}}^{2} + m_{2}^{2}} \right).
	\label{eq:sudakov_B_A}
\end{equation}
In this expression $q$ plays the role of a (virtual) loop momentum variable.

The Dirac delta enforces the $P = 0$ constraint. Enforcing this constraint leads to
\begin{equation}
	p_{34} = p_{1} - p_{3} = p_{4} - p_{2}, \qquad p_{31} = p_{1}, \qquad p_{42} = p_{2}.
\end{equation}
Thus,
\begin{equation}
	\abs{p_{34}}^{2} = -t, \qquad \abs{p_{31}}^{2} = -m_{1}^{2}, \qquad \abs{p_{42}}^{2} = -m_{2}^{2}.
\end{equation}
Let $k_{1}$ and $k_{2}$ be null spacetime vectors with units of mass. The Sudakov null decomposition of $p_{1}$ and $p_{2}$ is as follows:
\begin{equation}
	p_{1} = k_{1} + c_{12} k_{2}, \qquad p_{2} = k_{2} + c_{21} k_{1}.
\end{equation}
In terms of $p_{1}$ and $p_{2}$, you have
\begin{equation}
	k_{1} = \frac{p_{1} - c_{12} p_{2}}{1 - c_{12} c_{21}}, \qquad k_{2} = \frac{p_{2} - c_{21} p_{1}}{1 - c_{12} c_{21}}.
	\label{eq:k1_and_k2}
\end{equation}
From $|p_{1}|^{2} = -m_{1}^{2}$ and $|p_{2}|^{2} = -m_{2}^{2}$ it follows that
\begin{equation}
	c_{12} = - \frac{m_{1}^{2}}{2 (k_{1} \cdot k_{2})}, \qquad c_{21} = - \frac{m_{2}^{2}}{2 (k_{1} \cdot k_{2})} \quad \Longrightarrow \quad \frac{c_{12}}{c_{21}} = \frac{m_{1}^{2}}{m_{2}^{2}}.
\end{equation}
Using $\abs{k_{1}}^{2} = 0$ and $\abs{k_{2}}^{2} = 0$ you find quadratic equations for $c_{12}$ and $c_{21}$:
\begin{equation}
	m_{2}^{2} c_{12}^{2} + \left(m_{1}^{2} + m_{2}^{2} - s\right) c_{12} + m_{1}^{2} = 0, \qquad m_{1}^{2} c_{21}^{2} + \left(m_{1}^{2} + m_{2}^{2} - s\right) c_{21} + m_{2}^{2} = 0.
\end{equation}
Solving each quadratic equation yields
\begin{equation}
	c_{21} = \left( \frac{m_{2}^{2}}{m_{1}^{2}} \right) c_{12}, \qquad c_{12} = \frac{s - m_{1}^{2} - m_{2}^{2} \pm \sqrt{\Lambda(s)}}{2 m_{2}^{2}}.
\end{equation}
Note that $c_{12}$ and $c_{21}$ are (dimensionless) functions that can be written in terms of two (dimensionless) ratios
\begin{equation}
	\frac{s}{m_{1} m_{2}}, \qquad \frac{m_{1}}{m_{2}}.
\end{equation}
From $s = {-\abs{p_{1} + p_{2}}^{2}}$ it follows that
\begin{equation}
	2 (k_{1} \cdot k_{2}) = - \frac{s}{(1 + c_{12})(1 + c_{21})} = -m_{1} m_{2} \left[ \frac{2 m_{1} m_{2}}{s - m_{1}^{2} - m_{2}^{2} \pm \sqrt{\Lambda(s)}} \right].
\end{equation}
This can also be written as
\begin{equation}
	2 (k_{1} \cdot k_{2}) = -m_{1} m_{2} \left[ \frac{s - m_{1}^{2} - m_{2}^{2} \mp \sqrt{\Lambda(s)}}{2 m_{1} m_{2}} \right].
\end{equation}
Next you decompose the loop momentum $q$ as
\begin{equation}
	q = Q + a_{q} k_{1} + b_{q} k_{2}.
\end{equation}
Here $a_{q}$ and $b_{q}$ are Sudakov moduli. The integration measure over $q$ becomes
\begin{equation}
	\mathrm{d}q = \sqrt{\abs{k_{1}}^{2} \abs{k_{2}}^{2} - \left(k_{1} \cdot k_{2}\right)^{2}} \mathrm{d}a_{q} \mathrm{d}b_{q} \mathrm{d}Q.
\end{equation}
Note that the volume measure for $Q$ is in $D-2$ spacetime dimensions.

Now you write each of the factors in the denominator in (\ref{eq:sudakov_B_A}) in terms of the Sudakov moduli and the transversal momentum $Q$. First write
\begin{equation}
	p_{34} = P_{34} + a_{34} k_{1} + b_{34} k_{2}.
\end{equation}
Since $p_{34}$ is known, $a_{34}$, $b_{34}$, and $P_{34}$ are also known. From $k_{1} \cdot p_{34}$ and $k_{2} \cdot p_{34}$ it follows that
\begin{equation}
	b_{34} = \frac{k_{1} \cdot p_{34}}{k_{1} \cdot k_{2}}, \qquad a_{34} = \frac{k_{2} \cdot p_{34}}{k_{1} \cdot k_{2}}.
\end{equation}
Using (\ref{eq:k1_and_k2}) leads to:
\begin{equation}
	k_{1} \cdot p_{34} = \frac{(p_{1} - c_{12} p_{2}) \cdot (p_{1} - p_{3})}{1 - c_{12} c_{21}}, \qquad k_{2} \cdot p_{34} = \frac{(p_{2} - c_{21} p_{1}) \cdot (p_{1} - p_{3})}{1 - c_{12} c_{21}}.
\end{equation}
Recall that
\begin{align}
	s = -\abs{p_{1} + p_{2}}^{2} \quad &\Rightarrow \quad p_{1} \cdot p_{2} = \frac{m_{1}^{2} + m_{2}^{2} - s}{2}, \\
	t = -\abs{p_{1} - p_{3}}^{2} \quad &\Rightarrow \quad p_{1} \cdot p_{3} = \frac{t - 2m_{1}^{2}}{2}, \\
	u = -\abs{p_{2} - p_{3}}^{2} \quad &\Rightarrow \quad p_{2} \cdot p_{3} = \frac{u - m_{1}^{2} - m_{2}^{2}}{2}. \\
\end{align}
Thus,
\begin{equation}
	k_{1} \cdot p_{34} = \frac{t}{2} \left( \frac{1 + c_{12}}{1 - c_{12} c_{21}} \right), \qquad k_{2} \cdot p_{34} = -\frac{t}{2} \left( \frac{1 + c_{21}}{1 - c_{12} c_{21}} \right).
\end{equation}
Hence,
\begin{equation}
	a_{34} = \frac{t}{s} \left[ \frac{(1 + c_{12}) (1 + c_{21})^{2}}{1 - c_{12} c_{21}} \right], \qquad b_{34} = -\frac{t}{s} \left[ \frac{(1 + c_{12})^{2} (1 + c_{21})}{1 - c_{12} c_{21}} \right].
\end{equation}
Using $\abs{p_{12}}^{2} = -t$ it follows that
\begin{equation}
	\abs{P_{34}}^{2} = -t - 2 a_{34} b_{34} \left(k_{1} \cdot k_{2}\right) = -t - 2 \left[ \frac{\left(k_{1} \cdot p_{34}\right) \left(k_{2} \cdot p_{34}\right)}{\left(k_{1} \cdot k_{2}\right)} \right],
\end{equation}
which can be written as
\begin{equation}
	\abs{P_{34}}^{2} = -t \left( 1 + \frac{t}{s} \frac{(1 + c_{12})^{2}(1 + c_{21})^{2}}{(1 - c_{12} c_{21})^{2}} \right).
\end{equation}
The terms in the denominator in (\ref{eq:sudakov_B_A}) become:
\begin{align}
	\abs{q}^{2} &= \abs{Q}^{2} + 2 a_{q} b_{q} \left(k_{1} \cdot k_{2}\right), \\
	\abs{q - p_{31}}^{2} + m_{1}^{2} &= \abs{Q}^{2} + m_{1}^{2} + 2 \left(a_{q} + 1\right)\left(b_{q} + c_{12}\right)\left(k_{1} \cdot k_{2}\right), \\
	\abs{q - p_{34}}^{2} &= \abs{Q - P_{34}}^{2} + 2 \left(a_{q} - a_{34}\right)\left(b_{q} - b_{34}\right)\left(k_{1} \cdot k_{2}\right), \\
	\abs{q + p_{42}}^{2} + m_{2}^{2} &= \abs{Q}^{2} + m_{2}^{2} + 2 \left(a_{q} - c_{21}\right)\left(b_{q} - 1\right)\left(k_{1} \cdot k_{2}\right).
\end{align}
%%%%%%%%%%%%%%%%%%%%%%%%%%%%%%%%%%%%%%%%%%%%%%%%%%%%%%%%%%%%%%%%%%%%%%%%%%%%%%%%%%%%%%%%%%%%%%%%%%%%%%%%%%%%%%%%%%%%
\subsection{Feynman Moduli}
%%%%%%%%%%%%%%%%%%%%%%%%%%%%%%%%%%%%%%%%%%%%%%%%%%%%%%%%%%%%%%%%%%%%%%%%%%%%%%%%%%%%%%%%%%%%%%%%%%%%%%%%%%%%%%%%%%%%
After integrating over $q$ in (\ref{eq:hat_calB_A}), you find:
\begin{equation}
	\widehat{\mathcal{B}}_{A}(p) = \delta(P) \int\limits_{0}^{\infty} \int\limits_{0}^{\infty} \int\limits_{0}^{\infty} \int\limits_{0}^{\infty} \frac{\mathrm{d}T_{13} \mathrm{d}T_{21} \mathrm{d}T_{42} \mathrm{d}T_{34}}{(T_{13} + T_{21} + T_{42} + T_{34})^{D/2}} \exp{\left[ \frac{1}{2} \tilde{B}_{A}(p, T) \right]},
\end{equation}
where
\begin{equation}
	\tilde{B}_{A} = t T_{21} + \frac{|T_{13} p_{13} - T_{21} p_{21} - T_{42} p_{42}|^{2}}{T_{13} + T_{21} + T_{42} + T_{34}}.
\end{equation}
%%%%%%%%%%%%%%%%%%%%%%%%%%%%%%%%%%%%%%%%%%%%%%%%%%%%%%%%%%%%%%%%%%%%%%%%%%%%%%%%%%%%%%%%%%%%%%%%%%%%%%%%%%%%%%%%%%%%
\subsection{Regge Limit}
%%%%%%%%%%%%%%%%%%%%%%%%%%%%%%%%%%%%%%%%%%%%%%%%%%%%%%%%%%%%%%%%%%%%%%%%%%%%%%%%%%%%%%%%%%%%%%%%%%%%%%%%%%%%%%%%%%%%
...
%%%%%%%%%%%%%%%%%%%%%%%%%%%%%%%%%%%%%%%%%%%%%%%%%%%%%%%%%%%%%%%%%%%%%%%%%%%%%%%%%%%%%%%%%%%%%%%%%%%%%%%%%%%%%%%%%%%%
\subsection{Forward-JWKB Limit}
%%%%%%%%%%%%%%%%%%%%%%%%%%%%%%%%%%%%%%%%%%%%%%%%%%%%%%%%%%%%%%%%%%%%%%%%%%%%%%%%%%%%%%%%%%%%%%%%%%%%%%%%%%%%%%%%%%%%
...
%%%%%%%%%%%%%%%%%%%%%%%%%%%%%%%%%%%%%%%%%%%%%%%%%%%%%%%%%%%%%%%%%%%%%%%%%%%%%%%%%%%%%%%%%%%%%%%%%%%%%%%%%%%%%%%%%%%%
\section{Crossed Box}
%%%%%%%%%%%%%%%%%%%%%%%%%%%%%%%%%%%%%%%%%%%%%%%%%%%%%%%%%%%%%%%%%%%%%%%%%%%%%%%%%%%%%%%%%%%%%%%%%%%%%%%%%%%%%%%%%%%%
The cross-box correlator is given by
\begin{equation}
	\mathcal{C}_{A}(x) = G_{\Phi}(x_{1}| x_{3}) G_{A}(x_{4}| x_{1}) G_{\Psi}(x_{2}| x_{4}) G_{A}(x_{3}| x_{2}),
\end{equation}
but this expression is related to the box correlator (\ref{eq:box_A}) by swapping $x_{2} \longleftrightarrow x_{4}$.
%%%%%%%%%%%%%%%%%%%%%%%%%%%%%%%%%%%%%%%%%%%%%%%%%%%%%%%%%%%%%%%%%%%%%%%%%%%%%%%%%%%%%%%%%%%%%%%%%%%%%%%%%%%%%%%%%%%%
\subsection{Regge Limit}
%%%%%%%%%%%%%%%%%%%%%%%%%%%%%%%%%%%%%%%%%%%%%%%%%%%%%%%%%%%%%%%%%%%%%%%%%%%%%%%%%%%%%%%%%%%%%%%%%%%%%%%%%%%%%%%%%%%%
...
%%%%%%%%%%%%%%%%%%%%%%%%%%%%%%%%%%%%%%%%%%%%%%%%%%%%%%%%%%%%%%%%%%%%%%%%%%%%%%%%%%%%%%%%%%%%%%%%%%%%%%%%%%%%%%%%%%%%
\subsection{Forward-JWKB Limit}
%%%%%%%%%%%%%%%%%%%%%%%%%%%%%%%%%%%%%%%%%%%%%%%%%%%%%%%%%%%%%%%%%%%%%%%%%%%%%%%%%%%%%%%%%%%%%%%%%%%%%%%%%%%%%%%%%%%%
...
%%%%%%%%%%%%%%%%%%%%%%%%%%%%%%%%%%%%%%%%%%%%%%%%%%%%%%%%%%%%%%%%%%%%%%%%%%%%%%%%%%%%%%%%%%%%%%%%%%%%%%%%%%%%%%%%%%%%
\section{Vertex Corrections}
%%%%%%%%%%%%%%%%%%%%%%%%%%%%%%%%%%%%%%%%%%%%%%%%%%%%%%%%%%%%%%%%%%%%%%%%%%%%%%%%%%%%%%%%%%%%%%%%%%%%%%%%%%%%%%%%%%%%
There are two one-loop vertex corrections:
\begin{align}
	\mathcal{V}_{\Phi}(x) &= \delta(x_{2} - x_{4}) G_{A}(x_{1} | x_{3}) \int \mathrm{d}y \, G_{A}(y| x_{2}) G_{\Phi}(y| x_{1}) G_{\Phi}(y| x_{3}), \\
	\mathcal{V}_{\Psi}(x) &= \delta(x_{1} - x_{3}) G_{A}(x_{2} | x_{4}) \int \mathrm{d}y \, G_{A}(y| x_{1}) G_{\Psi}(y| x_{2}) G_{\Psi}(y| x_{4}).
\end{align}
%%%%%%%%%%%%%%%%%%%%%%%%%%%%%%%%%%%%%%%%%%%%%%%%%%%%%%%%%%%%%%%%%%%%%%%%%%%%%%%%%%%%%%%%%%%%%%%%%%%%%%%%%%%%%%%%%%%%
\subsection{Regge Limit}
%%%%%%%%%%%%%%%%%%%%%%%%%%%%%%%%%%%%%%%%%%%%%%%%%%%%%%%%%%%%%%%%%%%%%%%%%%%%%%%%%%%%%%%%%%%%%%%%%%%%%%%%%%%%%%%%%%%%
...
%%%%%%%%%%%%%%%%%%%%%%%%%%%%%%%%%%%%%%%%%%%%%%%%%%%%%%%%%%%%%%%%%%%%%%%%%%%%%%%%%%%%%%%%%%%%%%%%%%%%%%%%%%%%%%%%%%%%
\subsection{Forward-JWKB Limit}
%%%%%%%%%%%%%%%%%%%%%%%%%%%%%%%%%%%%%%%%%%%%%%%%%%%%%%%%%%%%%%%%%%%%%%%%%%%%%%%%%%%%%%%%%%%%%%%%%%%%%%%%%%%%%%%%%%%%
...
%%%%%%%%%%%%%%%%%%%%%%%%%%%%%%%%%%%%%%%%%%%%%%%%%%%%%%%%%%%%%%%%%%%%%%%%%%%%%%%%%%%%%%%%%%%%%%%%%%%%%%%%%%%%%%%%%%%%
\section{Vaccum Polarizations}
%%%%%%%%%%%%%%%%%%%%%%%%%%%%%%%%%%%%%%%%%%%%%%%%%%%%%%%%%%%%%%%%%%%%%%%%%%%%%%%%%%%%%%%%%%%%%%%%%%%%%%%%%%%%%%%%%%%%
There are two one-loop vacuum polarizations:
\begin{align}
	\mathcal{W}_{\Phi}(x) &= \delta(x_{1} - x_{3}) \delta(x_{2} - x_{4}) \int \int \mathrm{d}y_{1} \mathrm{d}y_{2} \, G_{A}(x_{1} | y_{1}) G_{A}(x_{2}| y_{2}) G_{\Phi}(y_{1} | y_{2}) G_{\Phi}(y_{2}| y_{1}), \\
	\mathcal{W}_{\Psi}(x) &= \delta(x_{1} - x_{3}) \delta(x_{2} - x_{4}) \int \int \mathrm{d}y_{1} \mathrm{d}y_{2} \, G_{A}(x_{1} | y_{1}) G_{A}(x_{2}| y_{2}) G_{\Psi}(y_{1} | y_{2}) G_{\Psi}(y_{2}| y_{1}).
\end{align}
%%%%%%%%%%%%%%%%%%%%%%%%%%%%%%%%%%%%%%%%%%%%%%%%%%%%%%%%%%%%%%%%%%%%%%%%%%%%%%%%%%%%%%%%%%%%%%%%%%%%%%%%%%%%%%%%%%%%
\subsection{Regge Limit}
%%%%%%%%%%%%%%%%%%%%%%%%%%%%%%%%%%%%%%%%%%%%%%%%%%%%%%%%%%%%%%%%%%%%%%%%%%%%%%%%%%%%%%%%%%%%%%%%%%%%%%%%%%%%%%%%%%%%
...
%%%%%%%%%%%%%%%%%%%%%%%%%%%%%%%%%%%%%%%%%%%%%%%%%%%%%%%%%%%%%%%%%%%%%%%%%%%%%%%%%%%%%%%%%%%%%%%%%%%%%%%%%%%%%%%%%%%%
\subsection{Forward-JWKB Limit}
%%%%%%%%%%%%%%%%%%%%%%%%%%%%%%%%%%%%%%%%%%%%%%%%%%%%%%%%%%%%%%%%%%%%%%%%%%%%%%%%%%%%%%%%%%%%%%%%%%%%%%%%%%%%%%%%%%%%
...