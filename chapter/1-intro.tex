\chapter{Introduction}
%%%%%%%%%%%%%%%%%%%%%%%%%%%%%%%%%%%%%%%%%%%%%%%%%%%%%%%%%%%%%%%%%%%%%%%%%%%%%%%%%%%%%%%%%%%%%%%%%%%%%%%%%%%%%%%%%%%%
...
%%%%%%%%%%%%%%%%%%%%%%%%%%%%%%%%%%%%%%%%%%%%%%%%%%%%%%%%%%%%%%%%%%%%%%%%%%%%%%%%%%%%%%%%%%%%%%%%%%%%%%%%%%%%%%%%%%%%
\section{Euler Gamma Function}
%%%%%%%%%%%%%%%%%%%%%%%%%%%%%%%%%%%%%%%%%%%%%%%%%%%%%%%%%%%%%%%%%%%%%%%%%%%%%%%%%%%%%%%%%%%%%%%%%%%%%%%%%%%%%%%%%%%%
The Euler Gamma function is
\begin{equation}
	\Gamma(z) = \int\limits_{0}^{\infty} \mathrm{d}x \left( \frac{1}{x} \right)^{1-z} \exp{(-x)}.
\end{equation}
Setting $x = \kappa^{2} w$ with $\kappa^{2} > 0$ leads to
\begin{equation}
	\Gamma(z) = \left( \kappa^{2} \right)^{z} \int\limits_{0}^{\infty} \mathrm{d}w \left( \frac{1}{w} \right)^{1-z} \exp{(- \kappa^{2} w)},
\end{equation}
which allows you to write
\begin{equation}
	\left( \frac{1}{\kappa^{2}} \right)^{z} = \frac{1}{\Gamma(z)} \int\limits_{0}^{\infty} \mathrm{d}w \left( \frac{1}{w} \right)^{1-z} \exp{(- \kappa^{2} w)}.
\end{equation}
Here $w$ is a Schwinger modulus.
%%%%%%%%%%%%%%%%%%%%%%%%%%%%%%%%%%%%%%%%%%%%%%%%%%%%%%%%%%%%%%%%%%%%%%%%%%%%%%%%%%%%%%%%%%%%%%%%%%%%%%%%%%%%%%%%%%%%
\section{Propagators}
%%%%%%%%%%%%%%%%%%%%%%%%%%%%%%%%%%%%%%%%%%%%%%%%%%%%%%%%%%%%%%%%%%%%%%%%%%%%%%%%%%%%%%%%%%%%%%%%%%%%%%%%%%%%%%%%%%%%
In the momentum basis, the propagator for a free quantum with mass $m$ is given by
\begin{equation}
	\widehat{G}_{m}(p, q) = \left( \frac{2}{\abs{p}^{2} + m^2} \right) \delta(p - q).
\end{equation}
Using a Schwinger modulus, this can be re-written as
\begin{equation}
	\widehat{G}_{m}(p, q) = \delta(p - q) \int\limits_{0}^{\infty} \mathrm{d}T \, \exp{\left[- \left( \frac{\abs{p}^2 + m^2}{2} \right) T\right]}.
\end{equation}
From the momentum basis, you can go to the position basis via a Fourier transform:
\begin{equation}
	G_{m}(x, y) = \int \int \mathrm{d}p \mathrm{d}q \, \widehat{G}_{m}(p, q) \exp{(i p \cdot x - i q \cdot y)}.
\end{equation}
Integration over $p$ and $q$ gives
\begin{equation}
	G_{m}(x, y) = \int\limits_{0}^{\infty} \mathrm{d}T \left( \frac{1}{T} \right)^{D/2} \exp{\left[- \frac{1}{2T}\abs{x - y}^{2} - \frac{1}{2} m^{2}T \right]}.
\end{equation}
As a special case, you can take the $m \rightarrow 0$ limit to obtain the propagator for a free massless quantum:
\begin{equation}
	G_{0}(x, y) = \int\limits_{0}^{\infty} \mathrm{d}T \left( \frac{1}{T} \right)^{D/2} \exp{\left[- \frac{1}{2T}\abs{x - y}^{2} \right]} = \left( \frac{2}{\abs{x - y}^{2}} \right)^{(D-2)/2} \Gamma\left( \frac{D - 2}{2} \right).
\end{equation}
This is valid as long as $D \neq 2$.
%%%%%%%%%%%%%%%%%%%%%%%%%%%%%%%%%%%%%%%%%%%%%%%%%%%%%%%%%%%%%%%%%%%%%%%%%%%%%%%%%%%%%%%%%%%%%%%%%%%%%%%%%%%%%%%%%%%%
\section{Kinematics}
%%%%%%%%%%%%%%%%%%%%%%%%%%%%%%%%%%%%%%%%%%%%%%%%%%%%%%%%%%%%%%%%%%%%%%%%%%%%%%%%%%%%%%%%%%%%%%%%%%%%%%%%%%%%%%%%%%%%
There are four external quanta; two incoming (labeled 1 and 2) and two outgoing (labeled 3 and 4). In the position basis, each external quantum is associated to a spacetime position. These four spacetime position vectors are independent. Similarly, in the momentum basis, each external quantum is associated to an energy-momentum vector that satisfies an on-shell constraint. Since this process is elastic, you have
\begin{equation}
	m_{1}^{2} = -\abs{p_{1}}^{2} = -\abs{p_{3}}^{2}, \qquad m_{2}^{2} = -\abs{p_{2}}^{2} = -\abs{p_{4}}^{2}.
\end{equation}
A priori, these four energy-momentum vectors are independent. But as you will see, due to translation invariance, the four energy-momentum vectors satisfy a linear constraint:
\begin{equation}
	p_{1} + p_{2} = p_{3} + p_{4}.
\end{equation}
There are three Mandelstam invariants:
\begin{equation}
	s = -\abs{p_{1} + p_{2}}^{2}, \qquad t = -\abs{p_{1} - p_{3}}^{2}, \qquad u = -\abs{p_{1} - p_{4}}^{2}.
\end{equation}
Due to the conservation constraint, it follows that
\begin{equation}
	s + t + u = 2 m_{1}^{2} + 2 m_{2}^{2}.
\end{equation}
An important function is
\begin{equation}
	\Lambda(s) = [s - (m_{1} - m_{2})^{2}] [s - (m_{1} + m_{2})^{2}].
\end{equation}
This is known as the K\"{a}ll\'{e}n function. Note that $\Lambda(s)$ can also be written as
\begin{equation}
	\Lambda(s) = (s - m_{1}^{2} - m_{2}^{2})^{2} - 4 m_{1}^{2} m_{2}^{2}.
\end{equation}
%%%%%%%%%%%%%%%%%%%%%%%%%%%%%%%%%%%%%%%%%%%%%%%%%%%%%%%%%%%%%%%%%%%%%%%%%%%%%%%%%%%%%%%%%%%%%%%%%%%%%%%%%%%%%%%%%%%%
\section{Change of Variables}
%%%%%%%%%%%%%%%%%%%%%%%%%%%%%%%%%%%%%%%%%%%%%%%%%%%%%%%%%%%%%%%%%%%%%%%%%%%%%%%%%%%%%%%%%%%%%%%%%%%%%%%%%%%%%%%%%%%%
In this section I will discuss different changes of position variables, and the corresponding conjugate momenta.
%%%%%%%%%%%%%%%%%%%%%%%%%%%%%%%%%%%%%%%%%%%%%%%%%%%%%%%%%%%%%%%%%%%%%%%%%%%%%%%%%%%%%%%%%%%%%%%%%%%%%%%%%%%%%%%%%%%%
\subsection{Incoming Basis}
%%%%%%%%%%%%%%%%%%%%%%%%%%%%%%%%%%%%%%%%%%%%%%%%%%%%%%%%%%%%%%%%%%%%%%%%%%%%%%%%%%%%%%%%%%%%%%%%%%%%%%%%%%%%%%%%%%%%
Consider the following expression:
\begin{equation}
	\mathbb{F} \equiv x_{1} \cdot p_{1} + x_{2} \cdot p_{2} - x_{3} \cdot p_{3} - x_{4} \cdot p_{4}.
	\label{eq:F}
\end{equation}
Now make the change of variables
\begin{equation}
	X \equiv \frac{x_{1} + x_{2} + x_{3} + x_{4}}{4}, \quad x_{34} \equiv x_{3} - x_{4}, \quad x_{31} \equiv x_{3} - x_{1}, \quad x_{42} \equiv x_{4} - x_{2}.
\end{equation}
The inverse relation is
\begin{align}
	x_{1} &= \frac{4 X + 2x_{34} - 3x_{31} + x_{42}}{4}, \\
	x_{2} &= \frac{4 X + 2x_{34} + x_{31} - 3x_{42}}{4}, \\
	x_{3} &= \frac{4 X + 2x_{34} + x_{31} + x_{42}}{4}, \\
	x_{4} &= \frac{4 X - 2x_{34} + x_{31} + x_{42}}{4}.
\end{align}
Then $\mathbb{F}$ can be written as
\begin{equation}
	\mathbb{F} \equiv X \cdot P + x_{34} \cdot p_{34} - x_{31} \cdot p_{31} - x_{42} \cdot p_{42},
\end{equation}
where
\begin{align}
	P &= p_{1} + p_{2} - p_{3} - p_{4}, \\
	p_{34} &= \frac{p_{1} - p_{2} - p_{3} + p_{4}}{2}, \\
	p_{31} &= \frac{3p_{1} - p_{2} + p_{3} + p_{4}}{4}, \\
	p_{42} &= \frac{-p_{1} + 3p_{2} + p_{3} + p_{4}}{4}.
\end{align}
%%%%%%%%%%%%%%%%%%%%%%%%%%%%%%%%%%%%%%%%%%%%%%%%%%%%%%%%%%%%%%%%%%%%%%%%%%%%%%%%%%%%%%%%%%%%%%%%%%%%%%%%%%%%%%%%%%%%
\subsection{Outgoing Basis}
%%%%%%%%%%%%%%%%%%%%%%%%%%%%%%%%%%%%%%%%%%%%%%%%%%%%%%%%%%%%%%%%%%%%%%%%%%%%%%%%%%%%%%%%%%%%%%%%%%%%%%%%%%%%%%%%%%%%
Consider (\ref{eq:F}) and make the change of variables
\begin{equation}
	X \equiv \frac{x_{1} + x_{2} + x_{3} + x_{4}}{4}, \quad x_{12} \equiv x_{1} - x_{2}, \quad x_{31} \equiv x_{3} - x_{1}, \quad x_{42} \equiv x_{4} - x_{2}.
\end{equation}
The inverse relation is
\begin{align}
	x_{1} &= \frac{4 X + 2x_{12} - x_{31} - x_{42}}{4}, \\
	x_{2} &= \frac{4 X - 2x_{12} - x_{31} - x_{42}}{4}, \\
	x_{3} &= \frac{4 X + 2x_{12} + 3x_{31} - x_{42}}{4}, \\
	x_{4} &= \frac{4 X - 2x_{12} - x_{31} + 3x_{42}}{4}.
\end{align}
Then $\mathbb{F}$ can be written as
\begin{equation}
	\mathbb{F} \equiv X \cdot P + x_{12} \cdot p_{12} - x_{31} \cdot p_{31} - x_{42} \cdot p_{42},
\end{equation}
where
\begin{align}
	P &= p_{1} + p_{2} - p_{3} - p_{4}, \\
	p_{12} &= \frac{p_{1} - p_{2} - p_{3} + p_{4}}{2}, \\
	p_{31} &= \frac{p_{1} + p_{2} + 3p_{3} - p_{4}}{4}, \\
	p_{42} &= \frac{p_{1} + p_{2} - p_{3} + 3p_{4}}{4}.
\end{align}
%%%%%%%%%%%%%%%%%%%%%%%%%%%%%%%%%%%%%%%%%%%%%%%%%%%%%%%%%%%%%%%%%%%%%%%%%%%%%%%%%%%%%%%%%%%%%%%%%%%%%%%%%%%%%%%%%%%%